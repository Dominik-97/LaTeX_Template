\section*{Úvod}
Pravdou je, že v moderním světě je mnoho problémů, které můžeme vnímat jako bezpečností hrozbu. Jednou z nejskloňovanějších hrozeb v očích odborníků jsou v moderní \textit{post-truth} době dezinformace, ty jsou v rámci momentálně stále probíhající světové pandemie silnou zbraní, prostřednictvím které je možné ovlivnit veřejné mínění, názory a v konečném důsledku tedy i samotné chování jednotlivých lidí. Mnoho subjektů, ať už se jedná o individua, větší skupiny, či dokonce národní/nadnárodní skupiny, které mohou být nezřídkakdy podporovány i státy, se prostřednictvím dezinformací snaží dosáhnout svých cílů nejen na scéně domácí, ale i v zahraničí. Považuji proto za velice důležité o dezinformacích mluvit a zasadit se za osvětovou kampaň, aby proti snaze o použití dezinformací lidé byli více imunní. I z toho důvodu jsem se rozhodl pojednat ve své seminární práci o tématu dezinformací, a to specificky ve světle momentální pandemické situace, práce je tedy zaměřena především na dezinformace spojené s pandemií koronaviru a snahou o jeho zvládnutí (Zde se dá hovořit například o dezinformace směřované vůči vakcínám.).

\section{Pojem dezinformace}

V moderní době internetové, kdy má přibližně 51\% obyvatel přístup k internetu\cite{noauthor_individuals_nodate}, se mohou dezinformace šířit stejnou rychlostí jako opravdivé informace a mnoho lidí tedy může snadlo podlehnout pocitu, že čtou pravdivé informace i když ve skutečnosti se jedná o dezinformace. Jde rovněž důvodně předpokládat, že v současné době, kdy mnoho lidí dma tráví více času než obvykle, bude dopad dezinformací značně vyšší.\\

Než ale přejdu k problematice dezinformací, je potřeba tento pojem nejdříve definovat. Pokud je třeba definovat jakýkoliv pojem, obracím se často na \textit{Oxford dictionary}, učinil jsem tak proto i teď, Oxfordský slovník tedy dezinformace definuje jako: \textit{"false information that is given deliberately"}\cite{noauthor_disinformation_nodate}, v překladu tedy jako \textit{"nepravdivá informace, která je sdělena záměrně}.

\section{Zdroje dezinformací}

\section{Jak proti dezinformacím bojovat?}

\section{Závěr}