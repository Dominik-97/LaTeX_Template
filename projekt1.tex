\section*{Úvod}
Pravdou je, že v moderním světě je mnoho problémů, které můžeme vnímat jako bezpečností hrozbu. Jednou z nejskloňovanějších hrozeb v očích odborníků jsou v moderní \textit{post-truth} době dezinformace, ty jsou v rámci momentálně stále probíhající světové pandemie silnou zbraní, prostřednictvím které je možné ovlivnit veřejné mínění, názory a v konečném důsledku tedy i samotné chování jednotlivých lidí. Mnoho subjektů, ať už se jedná o individua, větší skupiny, či dokonce národní/nadnárodní skupiny, které mohou být nezřídkakdy podporovány i státy, se prostřednictvím dezinformací snaží dosáhnout svých cílů nejen na scéně domácí, ale i v zahraničí. Považuji proto za velice důležité o dezinformacích mluvit a zasadit se za osvětovou kampaň, aby proti snaze o použití dezinformací lidé byli více imunní. I z toho důvodu jsem se rozhodl pojednat ve své seminární práci o tématu dezinformací, a to specificky ve světle momentální pandemické situace, práce je tedy zaměřena především na dezinformace spojené s pandemií koronaviru a snahou o jeho zvládnutí (Zde se dá hovořit například o dezinformace směřované vůči vakcínám.).

\section{Pojem dezinformace}

V moderní době internetové, kdy má přibližně 51\% obyvatel přístup k internetu\cite{noauthor_individuals_nodate}, se mohou dezinformace šířit stejnou rychlostí jako opravdivé informace a mnoho lidí tedy může snadlo podlehnout pocitu, že čtou pravdivé informace i když ve skutečnosti se jedná o dezinformace. Jde rovněž důvodně předpokládat, že v současné době, kdy mnoho lidí dma tráví více času než obvykle, bude dopad dezinformací značně vyšší.\\

Než ale přejdu k problematice dezinformací, je potřeba tento pojem nejdříve definovat. Pokud je třeba definovat jakýkoliv pojem, obracím se často na \textit{Oxford dictionary}, učinil jsem tak proto i teď, Oxfordský slovník tedy dezinformace definuje jako: \textit{"false information that is given deliberately"}\cite{noauthor_disinformation_nodate}, v překladu tedy jako \textit{"nepravdivá informace, která je sdělena záměrně}.\\

V této souvislosti je nutné zmínit dvě "podkategorie" dezinformací, a to výše definovanou dezinformaci jakožto úmyslně nepravdivou informaci a dále "misinformaci", či špatnou informaci, která je neúmyslně šířena i když je nepravdivá (chybí tedy aspekt úmyslu sdílet špatnou informaci), nicméně to nic nemění na faktu, že je stále sdílena nepravdivá informace, což v konečném důsledku může vést k úplně stejným výsledkům.\\

\subsection{Dezinformace v době Covidové}

Jednou z největších výzev v době probíhající pandemie COVID-19 byl, je a bude nejen boj proti koronaviru jako takovému, ale i proti dezinformacím, které jsou využíváný pro šíření nepravd či polopravd. Ve veřejném prostoru je možné sledovat mnoho diskuzí nad tím, jak se s dezinformacemi vypořádat.\\

Zdroje informací, a to jak v online prostoru, tak i v rámci televizního a rádiového vysílání, či dokonce v tisku jsou protkané nespočetným množstvím falešných a zavádějících tvrzeních týkajíce se zdroje, způsobu přenášení, závažnosti a v konečném důsledku i léčby virusu a snahy o jeho vymícení. Může se jednat o poměrně neškodná marketingová tvrzení ohledně zdánlivě účinných produktů jako například produkt od UVLEN \textregistered Technologies Korea\cite{uvlen__uvlen_nodate}, u kterého výrobce tvrdí, že má vlastnosti, které odporují základním zákonům fyziky (vlnovou délku světla není možné změnit za pomocí jednoduchého filtru), až po vážné útoky na vědce, a orgány veřejné moci, které mohou vyůstit až v občanskou neposlušnost, jako příklad je možné připomenout nedávný útok na Kapitol zfanatizovaných davem, který se nechal přesvědčit dezinformacemi vyslovenými americkým prezidentem Donaldem Trumpem.\\

V této souvislosti je problémem právě to, že některé z těchto misinformací jsou širší veřejností opravdu považovány za pravdivé.

\section{Zdroje dezinformací}

\section{Jak proti dezinformacím bojovat?}

\section{Závěr}